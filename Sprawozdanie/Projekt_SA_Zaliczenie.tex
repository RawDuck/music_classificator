\documentclass[10pt,a4paper]{article}
\usepackage{amsmath}
\usepackage{amsfonts}
\usepackage{amssymb}
\usepackage{graphicx}
\usepackage[polish]{babel}
\usepackage[utf8]{inputenc}
\usepackage{lmodern}
\selectlanguage{polish}
\usepackage{graphicx}
\usepackage{indentfirst}
\usepackage[T1]{fontenc}
\usepackage{multirow}
\usepackage{hyperref}
\usepackage{listings}
\usepackage{xcolor} 
\usepackage{caption}
\usepackage{subcaption}
\definecolor{Darkgreen}{rgb}{0,0.4,0}
\usepackage[T1]{fontenc}
\definecolor{listinggray}{gray}{0.9}
\definecolor{lbcolor}{rgb}{0.9,0.9,0.9}
\lstset{
backgroundcolor=\color{lbcolor},
    tabsize=4,    
%   rulecolor=,
    language=[GNU]C++,
        basicstyle=\scriptsize,
        upquote=true,
        aboveskip={1.5\baselineskip},
        columns=fixed,
        showstringspaces=false,
        extendedchars=false,
        breaklines=true,
        prebreak = \raisebox{0ex}[0ex][0ex]{\ensuremath{\hookleftarrow}},
        frame=single,
        numbers=left,
        showtabs=false,
        showspaces=false,
        showstringspaces=false,
        identifierstyle=\ttfamily,
        keywordstyle=\color[rgb]{0,0,1},
        commentstyle=\color[rgb]{0.026,0.112,0.095},
        stringstyle=\color[rgb]{0.627,0.126,0.941},
        numberstyle=\color[rgb]{0.205, 0.142, 0.73},
%        \lstdefinestyle{C++}{language=C++,style=numbers}’.
}
\lstset{
    backgroundcolor=\color{lbcolor},
    tabsize=4,
  language=C++,
  captionpos=b,
  tabsize=3,
  frame=lines,
  numbers=left,
  numberstyle=\tiny,
  numbersep=5pt,
  breaklines=true,
  showstringspaces=false,
  basicstyle=\footnotesize,
%  identifierstyle=\color{magenta},
  keywordstyle=\color[rgb]{0,0,1},
  commentstyle=\color{Darkgreen},
  stringstyle=\color{red}
  }
\author{Łukasz Smalec \\ Nr albumu: 39433\\
\\Robert Piątek \\ Nr albumu: 39402 }
\title{Sygnały akustyczne - Projekt zaliczeniowy}

\begin{document}
\maketitle
\newpage
\section{Założenia i realizacja projektu}
Projekt zakładał napisanie skryptu w języku \textit{Python}, pozwalający na dokonanie klasyfikacji utworu na podstawie jego gatunku. Jako metode klasyfikacji wykorzystano algorytm \textit{SVM} zaimplementowany z wykorzystaniem biblioteki \textit{sklearn}. Zrealizowany projekt pozwala na klasyfikacje utworu jako jeden z czterech gatunków: 
\begin{itemize}
\item Rock,
\item Pop,
\item Hiphop,
\item Metal.
\end{itemize}
Do nauki i testowania klasyfikatora wykorzystano 30-sekundowe fragmenty utworów ze zbioru danych \href{https://www.kaggle.com/andradaolteanu/gtzan-dataset-music-genre-classification}{GTZAN Dataset}. 

\section{Podział pracy}

\textbf{Łukasz Smalec}
\begin{itemize}
\item Znalezienie i przygotowanie zbioru danych.
\item Praca nad ekstraktorami cech.
\item Przygotowanie statystyk działania skryptu.
\item Zapis klasyfikatora i wyników do pliku.
\end{itemize}

\textbf{Robert Piątek}
\begin{itemize}
\item Znalezienie i przygotowanie zbioru uczącego.
\item Praca nad ekstraktorami cech.
\item Wstępne przetwarzanie danych.
\item Klasyfikacja.
\end{itemize}

\newpage

\section{Wyniki}
\subsection{Klasyfikator binarny}
Wstępnie zbudowano klasyfikator na 100 utworach z gatunku Rock oraz na mieszance 180 utworów o gatunkach innych niż Rock. Utwory zostały podzielone, 33$\%$ na zbiór testowy i 67$\%$ na zbiór uczący.
\begin{table}[h]
\centering
\caption{Wyniki klasyfikacji binarnej (czy utwór należy do gatunku Rock, czy nie).}
\label{tab:t1}
\begin{tabular}{cc}
\hline
\textbf{Metoda ekstrakcji cech} & \textbf{Poprawność klasyfikacji} \\ \hline
MFCC                            & 66.67\%                          \\
FFT                             & 78.49\%                          \\
Spectral Centroid               & 65.59\%                          \\ \hline
\end{tabular}
\end{table}

Wyniki przedstawione w tabeli \ref{tab:t1} wskazują, że najlepszą metodą jest  \textit{FFT}, zaś \textit{MFCC} i \textit{Spectral Centroid} zwróciły podobne wyniki trochę gorsze od \textit{FFT}. Należy jednak wspomnieć o tym, że \textit{FFT} wykonuje się wielokrotnie dłużej od poprzednich metod.

\subsection{Klasyfikator wieloklasowy --- pierwsza wersja}
Przy pierwszym podejściu w budowie klasyfikatora wieloklasowego wykorzystano 100 utworów z każdego z czterech wybranych gatunków. Utwory zostały podzielone tak samo jak wcześniej, 33$\%$ na zbiór testowy i 67$\%$ na zbiór uczący.
\begin{table}[h]
\centering
\caption{Wyniki klasyfikacji wieloklasowej.}
\label{tab:t2}
\begin{tabular}{ccc}
\hline
\textbf{Metoda ekstrakcji cech} & \textbf{Czas wykonania} & \textbf{Poprawność klasyfikacji} \\ \hline
MFCC                            & 20.50s                  & 66.23\%                          \\
FFT                             & 141.14s                 & 75.76\%                          \\
Spectral Centroid               & 36.52s                  & 36.36\%                          \\ \hline
\end{tabular}
\end{table}

\indent Wyniki ukazane w tabeli \ref{tab:t2} są zadowalające. Poziom poprawności klasyfikacji spadł jedynie dla ekstraktora \textit{Spectral Centroid}. Co ciekawe, \textit{FFT} utrzymało najlepszy wynik, jednak z czasem aż 7-krotnie dłuższym niż \textit{MFCC}. Z tego powodu zdecydowano się na dalsze badania w celu podniesienia poziomu klasyfikacji \textit{MFCC}.

\newpage
\subsection{Klasyfikator wieloklasowy --- badania nad podniesieniem dokładności \textit{MFCC}}
W celu podniesienia poprawności klasyfikacji dla cech MFCC, badano wpływ zmiany parametrów na wyniki. Zdecydowano się zmienić ilość generowanych cech, gdyż ten parametr miał największe znaczenie.
\begin{table}[h]
\centering
\caption{Wyniki badań w celu zwiększenia poprawności klasyfikacji na podstawie cech MFCC.}
\label{tab:t3}
\begin{tabular}{ccc}
\hline
\textbf{\begin{tabular}[c]{@{}c@{}}Ilość generowanych \\ cech MFCC\end{tabular}} & \textbf{Czas wykonania} & \textbf{Poprawność klasyfikacji} \\ \hline
25 000                                                                           & 20.93s                  & 66.67\%                          \\
50 000                                                                           & 22.07s                  & 68.94\%                          \\
100 000                                                                          & 23.64s                  & 72.73\%                          \\
150 000                                                                          & 26.43s                  & 71.97\%                          \\ \hline
\end{tabular}
\end{table}

Dzięki badaniom przedstawionym w tabeli \ref{tab:t3} udało się podnieść poprawność klasyfikacji do 72.73$\%$ przy 100 000 generowanych cech. Im więcej cech tym wyniki były lepsze, jednak próg 100 000 jest optymalny, gdyż po nim poprawność spada. Wynik jest bardzo zbliżony do poprawności \textit{FFT} --- 75.76$\%$. Biorąc pod uwagę zdecydowanie krótszy czas wykonania przy cechach \textit{MFCC}, wydaje się to lepsze rozwiązanie jeśli zależy na optymalności.
\newpage

\section{Szczegółowe wyniki klasyfikacji wieloklasowej}

\begin{table}[h]
\centering
\caption{Szczegółowe wyniki badań dla każdego gatunku.}
\label{tab:t4}
\begin{tabular}{cccc}
\hline
\textbf{Metoda}                                                              & \multicolumn{1}{l}{\textbf{Gatunek}} & \textbf{Poprawność klasyfikacji} & \textbf{Łączny czas wykonania} \\ \hline
\multirow{4}{*}{MFCC}                                                        & Rock                                 & 81.25\%                          & \multirow{4}{*}{24.07s}        \\
                                                                             & Hiphop                               & 48.72\%                          &                                \\
                                                                             & Pop                                  & 87.10\%                          &                                \\
                                                                             & Metal                                & 80.00\%                          &                                \\ \hline
\multirow{4}{*}{FFT}                                                         & Rock                                 & 81.25\%                          & \multirow{4}{*}{145.00s}       \\
                                                                             & Hiphop                               & 61.54\%                          &                                \\
                                                                             & Pop                                  & 80.65\%                          &                                \\
                                                                             & Metal                                & 83.33\%                          &                                \\ \hline
\multirow{4}{*}{\begin{tabular}[c]{@{}c@{}}Spectral\\ Centroid\end{tabular}} & Rock                                 & 43.75\%                          & \multirow{4}{*}{30.10s}        \\
                                                                             & Hiphop                               & 12.82\%                          &                                \\
                                                                             & Pop                                  & 32.26\%                          &                                \\
                                                                             & Metal                                & 63.33\%                          &                                \\ \hline
\end{tabular}
\end{table}

Jak można zauważyć w tabeli \ref{tab:t4} klasyfikator miał największy problem z gatunkiem Hiphop przy każdym rodzaju cech. Co ciekawe cechy \textit{MFCC} poradziły sobie najlepiej z gatunkiem Pop, zaś \textit{FFT} z gatunkiem Metal.

\section{Podsumowanie i przyszłość pracy}
Najlepszą cechą do klasyfikacji okazało się \textit{FFT} biorąc pod uwagę poprawność klasyfikacji, jednak pod względem optymalizacji klasyfikacji znacznie lepszą cechą jest \textit{MFCC}. Cecha \textit{Spectral Centroid} okazała się najgorsza, co zostało szczególnie mocno wykazane przy klasyfikacji wieloklasowej.\\
\indent Dalsze badania nad praca mogłyby doprowadzić do ciekawych wyników. Można przetestować różne inne klasyfikatory, czy też spróbować innych cech do ekstrakcji z utworów. Warto też przeprowadzić szczegółowe badania co do parametrów pozostałych ekstraktorów cech.







\end{document}
